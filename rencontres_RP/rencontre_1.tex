\documentclass{article}

\usepackage{geometry}
\usepackage[T1]{fontenc}
\usepackage[utf8]{inputenc}
\usepackage[francais]{babel}

\geometry{margin=2cm}

\begin{document}

\section*{PFA-Nethack : Compte-rendu de la première rencontre avec le Responsable pédagogique}


\subsection*{Organisation}
Il est prévu qu'il y ait des réunions d'environ une heure toutes les deux semaines.\\
Le responsable pédagogique a demandé que l'adresse du dépôt github lui soit envoyée, il a souligné le fait que compiler notre code devait être simple pour le client et le responsable pédagogique.\\
Le but étant d'arriver à un résultat propre, il nous a demandé de prévoir plusieurs phases de relecture ainsi qu'une composition du rapport au fil du projet afin que l'on puisse discuter fréquemment son contenu.\\
L'accent a aussi été mis sur le fait qu'il était nécessaire de planifier le développement.

\subsection*{Spécification des besoins}
Il est important que l'on détaille dans ce document ce que l'on pense faire, avec une liste des tâches qui composent l'ensemble du travail.\\
Il n'est pas nécessaire de présenter un diagramme de chaque type, le but sera de ne présenter que les diagrammes ayant réellement un intérêt.\\
Il sera attendu de nous que nos choix soient expliqués et justifiés.

\subsection*{Prochaine réunion}
Il a été défini que la prochaine réunion aurait lieu le mardi 6 novembre à 18h. Il faudra que l'on ait choisi le mode de développement, que l'on dispose d'un document avec des informations/questions écrites.

\end{document}
