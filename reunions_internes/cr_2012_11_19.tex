\documentclass{article}

\usepackage{geometry}
\usepackage[T1]{fontenc}
\usepackage[utf8]{inputenc}
\usepackage[francais]{babel}

\geometry{margin=2cm}


\newcommand {\ST}{Sven Taton}
\newcommand {\QL}{Qiwen Lu}
\newcommand {\LL}{Louis Leclec'h}
\newcommand {\DB}{David Bitonneau}
\newcommand {\AH}{Arthur Havlicek}
\newcommand {\BR}{Benoit Ruelle}
\newcommand {\LH}{Ludovic Hofer}

\begin{document}

\section*{PFA-Nethack : Compte-rendu de la réunion interne du 5 novembre 2012}

\subsection*{Présents}
\begin{itemize}
\item \LL
\item \BR
\item \AH
\item \ST
\item \DB
\item \LH
\end{itemize}

\subsection*{Tâches}
Un point a été fait sur les tâches qui étaient sensées être commencées,
avancées ou terminées pendant la semaine.

\subsubsection*{Recherche de portes : modifications du noyau}
Des patchs ont été produits sur certaines modifications du noyau, la
recherche de la désactivation des objets a été approfondie et il semble que
ce soit assez compliqué. La transmission de statistiques à la fin de la
partie reste à déclencher. La tâche est estimée accomplie à 30 \% (25 \%
lors de la dernière réunion).

\subsubsection*{Protocole de communication : interface et bot}
Le protocole est terminé dans l'ensemble, cependant il a été déterminé au
cours de la réunion qu'il est nécessaire d'ajouter la gestion d'une graine
aléatoire. La tâche est estimée accomplie à 100 \%, étant donné que les
ajouts et modifications restants ne sont que de légers ajustement.

\subsubsection*{I/O interface noyau}
Des tests ainsi que de la factorisation ont été effectués afin de détecter
plus facilement des problèmes silencieux ainsi que de simplifier le débogage.
Le traitement de certaines commandes a été ajouté, mais il en restes
certaines à implémenter. La tâche est estimée accomplie à 40 \% (25 \% lors
de la dernière réunion).

\subsubsection*{I/O interface bot}
Cette tâche n'a pas encore pu être commencée. Cependant étant donné que le
protocole est prêt, il est possible de commencer la tâche.

\subsubsection*{I/O interface BDD}
Le protocole n'est pas encore terminé, la tâche n'a donc pas pu être
commencée.

\subsubsection*{Sauvegarde de l'historique des parties}
Il a été déterminé qu'il suffisait de conserver la graine utilisée par le
noyau pour assurer que celui-ci ait un comportement déterministe. Il a été
déterminé au cours de la réunion que l'interface fournirait une graine
aléatoire au bot et que ce serait à celui-ci d'assurer un comportement
déterministe en fonction de cette graine.

\subsubsection*{Protocole : interface BDD}
Il a été déterminé que nous utiliserions du xml en transition entre
l'interface et la base de données, permettant ainsi de créer de nouvelles
tables à la volée. Un exemple est déjà fourni, le formalisme reste à faire.
La tâche est estimée accomplie à 60 \% (30 \% lors de la dernière réunion).

\subsubsection*{Starter Package : mode recherche de portes}
Une version java nécessitant encore quelques modifications légères (lecture
depuis une socket à la place de lecture depuis l'entrée standard et prise en
charge de la graine envoyée par l'interface). La tâche est estimée accomplie
à 50 \% (Deux starters packages étant requis).

\subsubsection*{Outils Statistique}
La tâche n'a pas été commencée.

\subsubsection*{Protocole de communication : Base de données et affichage de
statistiques}
Le protocole a été débuté mais il doit encore être mis sur le wiki. La tâche
est estimée accomplie à 10 \%.

\subsubsection*{Affichage des statistiques}
La tâche n'a pas été commencée.

\subsubsection*{Outil de replay}
Le principe a été validé, mais l'implémentation doit encore être faîte. La
tâche est estimée accomplie à 15 \%.

\subsubsection*{Protocole : Historique de partie}
Une idée assez précise semble fonctionner, mais rien n'a encore été rédigé.
La tâche est estimée accomplie à 15 \%.

\subsection*{Gestion du travail en équipe}
Certaines personnes qui n'avaient pas encore travaillé sur le sujet ont
commencé à s'y mettre, cependant, leurs avancées n'ont pas été mises à
disposition des autres, l'accent a donc été mis sur le fait qu'il était
nécessaire de mettre les avancées en ligne.\\
Différents problèmes ont été évoqués par différentes personnes. Afin
d'assurer la meilleur entente possible dans le groupe, ainsi qu'une
participation possible pour chacun, on a tenté d'y trouver des solutions.\\
Pour certains, une frustration est ressentie car leurs idées et leur travail
n'ont pas été retenus, l'importance de s'accorder sur des points pouvant
être litigieux avant de commencer à coder à donc été soulignée, afin que ce
type de problème ne se répète pas.\\
Un autre problème soulevé est le fait que certains se sentent un peu dépassés
par les termes ou encore les outils utilisés. Il leur a donc été signalé
qu'il ne fallait pas qu'ils hésitent à poser leurs questions par mail afin
qu'ils ne restent pas bloqués.\\
Il y a aussi un problème d'approche, car certains ne sont
pas vraiment familier avec le développement en parallèle et ont tendance à
attendre que les autres parties soient finis avant de se mettre à la leur.
Des explications ont été données pour plus facilement parvenir à travailler
en équipe.\\
Un dernier problème évoqué est un malaise pour certains qui n'ont pas envie
de travailler et que certains décident ensuite si leur travail convient. Ils
préfèrent avoir l'impression qu'on leur propose des choses plutôt que de se
sentir supervisés. La remarque a été prise en compte, tout en émettant la
réserve que ce n'était pas toujours aux mêmes de proposer et qu'il était
nécessaire que chacun puisse avancer dans son travail seul en sollicitant de
l'aide s'il est bloqué.

\subsection*{Plan du document de spécification}
Le plan du document de spécification a été relu en groupe et certaines
modifications y ont été apportées.

\subsection*{Prochaine réunion}
Il a été décidé que la prochaine réunion aurait lieu le lundi 26 novembre à 13h30.

\end{document}
