\documentclass{article}

\usepackage{geometry}
\usepackage[T1]{fontenc}
\usepackage[utf8]{inputenc}
\usepackage[francais]{babel}

\geometry{margin=2cm}


\newcommand {\ST}{Sven Taton}
\newcommand {\QL}{Qiwen Lu}
\newcommand {\LL}{Louis Leclec'h}
\newcommand {\DB}{David Bitonneau}
\newcommand {\AH}{Arthur Havlicek}
\newcommand {\BR}{Benoit Ruelle}
\newcommand {\LH}{Ludovic Hofer}

\begin{document}

\section*{PFA-Nethack : Compte-rendu de la réunion interne du 5 novembre 2012}

\subsection*{Présents}
\begin{itemize}
\item \ST
\item \LL
\item \DB
\item \AH
\item \BR
\item \LH
\end{itemize}

\subsection*{Tâches}
En accord commun, certaines tâches oubliées ont été ajoutées à la liste, principalement concernant l'historique des parties et le replay.\\
La planification à l'aide de planner a été acceptée par les présents et il a été décidé que les estimations utilisées seraient décrites en nombre de jours nécessaires pour réaliser la tâche seul.\\
Les estimations ont été revues ensemble afin que tout le monde les approuve.\\
Des tâches ont été attribuées pour une durée d'environ 1 mois.

\subsection*{Stockage des statistiques}
Il a été décidé à l'unanimité des présents que nous utiliserions une base de données pour stocker le résultat des parties qui nous serviront à faire des statistiques.

\subsection*{Méthode de développement}
Après discussion, la méthode Scrum n'a pas été retenue, car nous considérons qu'elle est peu adaptée à notre projet. Nous avons décidé de nous servir de la planification que nous avons effectuée pour vérifier une fois par semaine que l'avancement du projet est conforme à ce qui a été prévu, sans pour autant nous organiser en sprint ou changer les objectifs à chaque fois, la dépendance entre nos différentes tâches étant très forte.

\subsection*{Méthode de création des modes}
Il a été décidé que nous partions sur un principe de plusieurs patchs par mode, un mode étant en fait composé de plusieurs patchs à appliquer. Cependant, si nous avons assez de temps à la fin, il est envisagé de présenter une autre version contenant un seul patch qui permettra de choisir le mode à l'exécution du programme.

\subsection*{Méthode de déploiement et gestion git}
Il a été décidé à l'unanimité des présents qu'aucune version modifiée du code source de nethack serait présente dans la branche master. Celle-ci contiendra plutôt des patchs ainsi que des scripts correspondant à des ensemble de patchs. De plus, il a été décidé que personne n'aurait le droit de pusher ses propres modifications dans master, elles devront forcément être validée par une autre personne.

\subsection*{Ordre de développement}
Il a été décidé que nous avancerions sur un seul mode dans un premier temps. Dès que celui-ci sera parfaitement opérationnel, nous pourrons ajouter de nouveaux modes et éventuellement en développer en parallèle. Ce choix a été pris afin de ne pas risquer d'avoir aucun mode opérationnel à l'approche de la deadline. Il permet aussi d'avoir plus de marge pour corriger les outils principaux si une des approche a été mal choisie.

\end{document}
