\documentclass{article}

\usepackage{geometry}
\usepackage[T1]{fontenc}
\usepackage[utf8]{inputenc}
\usepackage[francais]{babel}

\geometry{margin=2cm}


\newcommand {\ST}{Sven Taton}
\newcommand {\QL}{Qiwen Lu}
\newcommand {\LL}{Louis Leclec'h}
\newcommand {\DB}{David Bitonneau}
\newcommand {\AH}{Arthur Havlicek}
\newcommand {\BR}{Benoit Ruelle}
\newcommand {\LH}{Ludovic Hofer}
\newcommand {\accomplished}[2]{La tâche est estimée accomplie à #1 \%
(#2 \% lors de la dernière réunion).}

\begin{document}

\section*{PFA-Nethack : Compte-rendu de la réunion interne du 5 novembre 2012}

\subsection*{Présents}
\begin{itemize}
\item \LL
\item \BR
\item \AH
\item \QL
\item \DB
\item \LH
\end{itemize}

\subsection*{Tâches}
Un point a été fait sur les tâches qui étaient sensées être commencées,
avancées ou terminées pendant la semaine.

\subsubsection*{Recherche de portes : modifications du noyau}
Des recherches ont été effectuées sur un moyen de connaître le nombre
de portes secrètes. Il y a un accès à un tableau correspondant à la map avec
entre autre un accès au type de la case. Deux types différents ont été trouvés,
portes secrètes et corridors secrets, il faut déterminer si l'on peut
considérer que les deux sont du même type ou s'il faut les différencier.
La communication de ce type de statistiques se fera par un fichier de log lu
ensuite par le simulateur de partie. La partie de code correspondant à cette
production n'est pas encore écrite, mais le principe est validé.
La suppression des objets ne fonctionne pas pour l'instant.
Le problème de fichier manquant lors du retrait des pièges est encore présent,
mais uniquement assez tard dans la partie.
\accomplished{65}{30}
La tâche est estimée accomplie à 65 \% (30 \% lors de la dernière réunion).

\subsubsection*{I/O interface noyau}
Il est encore nécessaire de raffiner la documentation et d'établir certaines
fonctions d'envoi de message au noyau.
\accomplished{70}{40}

\subsubsection*{I/O interface bot}
Le côté transmission de message du serveur a été préparé, il reste à faire la
traduction des commandes reçues ainsi que leur validation.
\accomplished{10}{0}

\subsubsection*{I/O interface BDD}
Le terrain a été préparé, mais le protocole reste à finaliser.
Il est possible de valider la sortie du noyau par le xsd afin de vérifier si
elle est valide.
Les avancées majeures ont été du côté de la base de données.
\accomplished{25}{0}


\subsubsection*{Sauvegarde de l'historique des parties}
Cette avancée est un peu bloquée car il est nécessaire de modifier le code du
noyau et de l'interface pour faire tourner l'historique.
\accomplished{50}{25}


\subsubsection*{Protocole : interface BDD}
Il est encore nécessaire de définir le xsd.
\accomplished{80}{60}


\subsubsection*{Starter Package : mode recherche de portes}
La compatibilité avec la transmission par des socket a été assurée et des
tests ont été ajoutés.
\accomplished{50}{50}

\subsubsection*{Outils Statistique}
Des choix ont été fait concernant la bibliothèque.
\accomplished{30}{0}

\subsubsection*{Protocole de communication : Base de données et affichage de
statistiques}
Il y a un désaccord entre les deux personnes chargées du protocole, apparemment
la version en ligne ne correspond pas à celle qui va être implémentée.
\accomplished{10}{10}

\subsubsection*{Affichage des statistiques}
Quelques réflexions ont eu lieu au sujet de l'affichage souhaité ainsi que des
moyens pour y parvenir, mais rien n'a encore été concrètement commencé.
\accomplished{0}{0}

\subsubsection*{Outil de replay}
L'avancée est bloquée par une attente de la fin de l'interface.
\accomplished{15}{15}

\subsubsection*{Protocole : Historique de partie}
La doc spécifique reste toujours à faire, quelques ajouts ont été faits sur
d'autres wiki, mais il est intéressant d'avoir toutes les informations ayant
trait à l'historique regroupée dans un seul protocole.
\accomplished{25}{15}


\subsection*{Document de spécification}
Étant donné qu'il n'y avait pas eu beaucoup d'avancement au cours de la semaine,
un petit rush a été organisé ou chacun avait une partie à rédiger, puis le
document a été relu et validé en groupe.

\subsection*{Prochaine réunion}
Il a été décidé que la prochaine réunion aurait lieu le lundi 3 décembre à
13h30.

\end{document}
